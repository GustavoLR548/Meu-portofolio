\documentclass[aspectratio=169]{beamer}	 	
% ---
% PACOTES
% ---
\usepackage[alf]{abntex2cite}	% Citações padrão ABNT
\usepackage[brazil]{babel}		% Idioma do documento
\usepackage{color}			      % Controle das cores
\usepackage{xcolor}
\usepackage[T1]{fontenc}		  % Selecao de codigos de fonte.
\usepackage{graphicx}			    % Inclusão de gráficos
\usepackage[utf8]{inputenc}		% Codificacao do documento (conversão automática dos acentos)
\usepackage{txfonts}			    % Fontes virtuais
\usepackage{amsmath}
\usepackage{tikz}
\usetikzlibrary{arrows.meta}

\usetheme{Warsaw}

\setbeamercolor{normal text}{fg=white,bg=black!85}
\setbeamercolor{structure}{fg=white}

\setbeamercolor{alerted text}{fg=red!85!black}

\setbeamercolor{item projected}{use=item,fg=black,bg=item.fg!35}

\setbeamercolor*{framesubtitle}{fg=white}

\setbeamercolor*{block title}{parent=structure,bg=black!60}
\setbeamercolor*{block body}{fg=black,bg=black!10}
\setbeamercolor*{block title alerted}{parent=alerted text,bg=black!15}
\setbeamercolor*{block title example}{parent=example text,bg=black!15}

\setbeamercolor{titlelike}{parent=structure}

% para fontes matemáticas
% Enconte mais temas e cores em http://www.hartwork.org/beamer-theme-matrix/ 
% Veja também http://deic.uab.es/~iblanes/beamer_gallery/index.html

% Customizações de Cores: fg significa cor do texto e bg é cor do fundo

% informações do PDF
\makeatletter
\hypersetup{
     	%pagebackref=true,
		pdftitle={\@title}, 
		pdfauthor={\@author},
    	pdfsubject={Hands-On 14},
	    pdfcreator={GLR},
		pdfkeywords={HandsOn14}{LaTeX}, 
		colorlinks=true,       		% false: boxed links; true: colored links
    	linkcolor=black,          	% color of internal links
    	citecolor=blue,        		% color of links to bibliography
    	filecolor=magenta,      		% color of file links
		urlcolor=blue,
		bookmarksdepth=4
}
\makeatother

    %TABELA DE CONTEÚDO

    \setbeamercolor{section in toc}{fg=alerted text.fg, fg = red}
    

    \definecolor{SectionBox}{RGB}{60,160,0}
    \definecolor{UniGreen}{RGB}{ 0,139,0}
    \definecolor{TitleGreen}{RGB}{0,120,0}
    \setbeamercolor{title}{fg=TitleGreen}
    \setbeamercolor{structure}{fg=UniGreen} 
    \setbeamercolor{item projected}{bg=UniGreen, fg=white} %define color of the item-bullets
    \setbeamercolor{button}{bg=UniGreen}

    \setbeamertemplate{section in head/foot}{\hfill\protect\tikz\protect\node[rectangle,fill=SectionBox!90,rounded corners=1pt,inner sep=1pt,] {\textcolor{white}{\insertsectionhead}};}
    \setbeamertemplate{section in head/foot shaded}{\textcolor{structure!40}{\hfill\insertsectionhead}}

    \setbeamertemplate{headline}{%
\leavevmode%
  \hbox{%
    \begin{beamercolorbox}[wd=\paperwidth,ht=2.5ex,dp=1.125ex]{palette quaternary}%
    \insertsectionnavigationhorizontal{\paperwidth}{}{\hskip0pt plus1filll}
    \end{beamercolorbox}%
  }
}

\insertsectionnavigationhorizontal{\paperwidth}{\hskip0pt plus1filll}{\hskip0pt plus1filll}


% ---

% --- Informações do documento ---
\title{Hands-On 14}
% ---

% ----------------- INÍCIO DO DOCUMENTO --------------------------------------
\begin{document}
    % ----------------- NOVO SLIDE --------------------------------
    \begin{frame}

    \titlepage

    \end{frame}

    % ----------------- NOVO SLIDE --------------------------------]
    \section{Perguntas}

    \begin{frame}{Perguntas}


        \textbf{1.} O escalonamento $S_a$ é completo? Justifique sua resposta.

        \textbf{2.} Considerando que as últimas operações no escalonamento $S_a$ 
        sejam $c_2$, $c_3$, $c_1$, nessa ordem, o escalonamento $S_a$ é recuperável? 
        Justifique sua resposta apresentando todas as leituras sujas 
        existentes.

        \textbf{3.} O escalonamento $S_a$ é serializável? Justifique sua resposta 
        apresentando o grafo de precedência completo.


    \end{frame}
    % ----------------- NOVO SLIDE --------------------------------
    \begin{frame}{Perguntas}

        Considere que o escalonamento Sa apresentado abaixo foi constituído
        a partir das transações $T_1$, $T_2$ e $T_3$ também apresentadas abaixo. 
        Ressalta-se que, em um SGBDR diversas transações devem ser escalonadas
        para executarem simultaneamente, aumentando assim a concorrência e,
        consequentemente, diminuindo o tempo de processamento. No entanto, 
        tal concorrência demanda a utilização de técnicas de controle de 
        concorrência para garantir as propriedades de Atomicidade, 
        Consistência, Isolamento e Durabilidade (ACID). \newline

            $T_1$ = r(x), r(y), w(x), r(z)

            $T_2$ = r(z), r(x), r(y), w(z)

            $T_3$ = r(y), r(z), w(y), r(x)

            $S_a$ = $r_3$(y), $r_2$(z), $r_1$(x), $r_2$(x), $r_3$(z), $r_2$(y), $w_3$(y),
            $r_1$(y), $w_2$(z), $w_1$(x), $r_3$(x), $r_1$(z)

        

    \end{frame}
    % ----------------- NOVO SLIDE --------------------------------
    \section{Questão1}

    \begin{frame}{Questão1}

        R: O escalonamento $S_a$ não é completa, pois ela não atende 
        os requisitos para ser um escalonamento completo. O mais perceptível, é que este 
        não apresenta a ordem das operações nas transações originais preservada.

    \end{frame}

    \section{Questão2}

    \begin{frame}{Questão2}

        R: O escalonamento $S_a$ não é recuperável, devido a presença de leituras sujas: \newline

        $S_a$ = $r_3$(y), $r_2$(z), $r_1$(x), $r_2$(x), $r_3$(z), \colorbox{red}{$r_2$(y)}, \colorbox{red}{$w_3$(y)},
        $r_1$(y), $w_2$(z), $w_1$(x), $r_3$(x), $r_1$(z)

    \end{frame}

    \section{Questão3}

    \begin{frame}{Questão3}

        \begin{figure}[ht]
            \centering
            \begin{tikzpicture}[main/.style = {draw, circle,minimum size=10mm}] 
              \node[main] (1) at (2,2)  {$T_1$}; 
              \node[main] (2) at (0,2)  {$T_2$}; 
              \node[main] (3) at (-2,0) {$T_3$};
        
              \draw [-{Latex[length=3mm]}] (2) to [out=150,in=130] (3);
              \draw [-{Latex[length=3mm]}] (3) to [out=-50,in=-50] (2);
              \draw [-{Latex[length=3mm]}] (2) -- (1);
            \end{tikzpicture} 
            \caption{Grafo de precedência para $S_a$}
          \end{figure}
          Como demonstrando pela figura acima, o escalonamento $S_a$ não é 
          serializável, pois existe um ciclo entre $T_3$ e $T_2$.
    \end{frame}
% ----------------- FIM DO DOCUMENTO -----------------------------------------
\end{document}