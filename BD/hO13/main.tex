\documentclass[aspectratio=169]{beamer}	 	
% ---
% PACOTES
% ---
\usepackage[alf]{abntex2cite}	% Citações padrão ABNT
\usepackage[brazil]{babel}		% Idioma do documento
\usepackage{color}			      % Controle das cores
\usepackage[T1]{fontenc}		  % Selecao de codigos de fonte.
\usepackage{graphicx}			    % Inclusão de gráficos
\usepackage[utf8]{inputenc}		% Codificacao do documento (conversão automática dos acentos)
\usepackage{txfonts}			    % Fontes virtuais
\usepackage{amsmath}
\usepackage{tikz}
\usetikzlibrary{positioning}

\usetheme{Warsaw}

\setbeamercolor{normal text}{fg=white,bg=black!85}
\setbeamercolor{structure}{fg=white}

\setbeamercolor{alerted text}{fg=red!85!black}

\setbeamercolor{item projected}{use=item,fg=black,bg=item.fg!35}

\setbeamercolor*{framesubtitle}{fg=white}

\setbeamercolor*{block title}{parent=structure,bg=black!60}
\setbeamercolor*{block body}{fg=black,bg=black!10}
\setbeamercolor*{block title alerted}{parent=alerted text,bg=black!15}
\setbeamercolor*{block title example}{parent=example text,bg=black!15}

\setbeamercolor{titlelike}{parent=structure}

% para fontes matemáticas
% Enconte mais temas e cores em http://www.hartwork.org/beamer-theme-matrix/ 
% Veja também http://deic.uab.es/~iblanes/beamer_gallery/index.html

% Customizações de Cores: fg significa cor do texto e bg é cor do fundo

% informações do PDF
\makeatletter
\hypersetup{
     	%pagebackref=true,
		pdftitle={\@title}, 
		pdfauthor={\@author},
    	pdfsubject={Hands-On 13},
	    pdfcreator={GLR},
		pdfkeywords={HandsOn13}{LaTeX}, 
		colorlinks=true,       		% false: boxed links; true: colored links
    	linkcolor=black,          	% color of internal links
    	citecolor=blue,        		% color of links to bibliography
    	filecolor=magenta,      		% color of file links
		urlcolor=blue,
		bookmarksdepth=4
}
\makeatother

    %TABELA DE CONTEÚDO

    \setbeamercolor{section in toc}{fg=alerted text.fg, fg = red}
    

    \definecolor{SectionBox}{RGB}{60,160,0}
    \definecolor{UniGreen}{RGB}{ 0,139,0}
    \definecolor{TitleGreen}{RGB}{0,120,0}
    \setbeamercolor{title}{fg=TitleGreen}
    \setbeamercolor{structure}{fg=UniGreen} 
    \setbeamercolor{item projected}{bg=UniGreen, fg=white} %define color of the item-bullets
    \setbeamercolor{button}{bg=UniGreen}

    \setbeamertemplate{section in head/foot}{\hfill\protect\tikz\protect\node[rectangle,fill=SectionBox!90,rounded corners=1pt,inner sep=1pt,] {\textcolor{white}{\insertsectionhead}};}
    \setbeamertemplate{section in head/foot shaded}{\textcolor{structure!40}{\hfill\insertsectionhead}}

    \setbeamertemplate{headline}{%
\leavevmode%
  \hbox{%
    \begin{beamercolorbox}[wd=\paperwidth,ht=2.5ex,dp=1.125ex]{palette quaternary}%
    \insertsectionnavigationhorizontal{\paperwidth}{}{\hskip0pt plus1filll}
    \end{beamercolorbox}%
  }
}

\insertsectionnavigationhorizontal{\paperwidth}{\hskip0pt plus1filll}{\hskip0pt plus1filll}


% ---

% --- Informações do documento ---
\title{Hands-On 13}
% ---

% ----------------- INÍCIO DO DOCUMENTO --------------------------------------
\begin{document}
    % ----------------- NOVO SLIDE --------------------------------
    \begin{frame}

    \titlepage

    \end{frame}

    % ----------------- NOVO SLIDE --------------------------------]
    \section{Query 1}

    \begin{frame}{Query}

      SELECT DISTINCT A.CPF, A.Nome

      FROM Funcionarios A

      WHERE A.CPF IN (SELECT CPF\_Supervisor FROM Funcionarios)

      AND A.CPF NOT IN (SELECT CPF FROM Clientes)

    \end{frame}
    \begin{frame}{Query Otimizada}
      SELECT DISTINCT CPF, CPF\_Supervisor INTO TMP1

      FROM Funcionarios;

      SELECT DISTINCT A.CPF, A.Nome

      FROM Funcionarios A, TMP

      LEFT OUTER JOIN Clientes C ON A.CPF = C.CPF WHERE C.CPF IS NULL
    \end{frame}
    % ----------------- NOVO SLIDE --------------------------------
    \section{Query 2}

    \begin{frame}{Query}

        SELECT A.CodFilme, B.Nome

        FROM Midias A, Filmes B

        WHERE A.CodFilme = B.Codigo

        AND A.Tipo = 'DVD'

        OR A.Tipo = 'VHS'

    \end{frame}
    \begin{frame}{Query Otimizada}

        SELECT A.CodFilme, B.Nome

        FROM Midias A, Filmes B

        WHERE A.CodFilme = B.Codigo AND (A.Tipo = 'DVD')

        UNION

        SELECT A.CodFilme, B.Nome

        FROM Midias A, Filmes B

        WHERE A.CodFilme = B.Codigo AND (A.Tipo = 'VHS')

    \end{frame}
    % ----------------- NOVO SLIDE --------------------------------
    \section{Query 3}

    \begin{frame}{Query}

        SELECT A.CPF\_Cliente, A.ID\_Midia, A.DataLocacao

        FROM Aluguel A, Clientes B

        WHERE A.CPF\_Cliente = B.CPF

        AND B.Sexo != "F"

        AND EXISTS (SELECT * FROM Pagamentos C

        WHERE C.CPF\_Cliente = A.CPF\_Cliente

        AND C.ID\_Midia = A.ID\_Midia

        AND C.DataLocacao = A.DataLocacao)

    \end{frame}

    \begin{frame}{Query Otimizada}

        SELECT * INTO TEMP
        
        FROM Pagamentos 

        GROUP BY CPF\_Cliente, ID\_Midia, DataLocacao;

        SELECT DISTINCT A.CPF\_Cliente, A.ID\_Midia, A.DataLocacao

        FROM Aluguel A, Clientes B, TEMP C

        WHERE A.CPF\_Cliente = B.CPF

        AND B.Sexo != "F"

    \end{frame}

% ----------------- FIM DO DOCUMENTO -----------------------------------------
\end{document}
