\documentclass[aspectratio=169]{beamer}	 	

\usetheme{Rochester}
\usecolortheme{crane}
\usefonttheme{professionalfonts}		
\usepackage{xcolor}

\pagecolor[rgb]{0.5,0.5,0.5}
\color[rgb]{1,1,1}

% para fontes matemáticas
% Enconte mais temas e cores em http://www.hartwork.org/beamer-theme-matrix/ 
% Veja também http://deic.uab.es/~iblanes/beamer_gallery/index.html

% Customizações de Cores: fg significa cor do texto e bg é cor do fundo

% informações do PDF
\makeatletter
\hypersetup{
     	%pagebackref=true,
		pdftitle={\@title}, 
		pdfauthor={\@author},
    	pdfsubject={Hands-On 12},
	    pdfcreator={GLR},
		pdfkeywords={HandsOn12}{LaTeX}, 
		colorlinks=true,       		% false: boxed links; true: colored links
    	linkcolor=black,          	% color of internal links
    	citecolor=blue,        		% color of links to bibliography
    	filecolor=magenta,      		% color of file links
		urlcolor=blue,
		bookmarksdepth=4
}
\makeatother

% ---
% PACOTES
% ---
\usepackage[alf]{abntex2cite}	% Citações padrão ABNT
\usepackage[brazil]{babel}		% Idioma do documento
\usepackage{color}			      % Controle das cores
\usepackage[T1]{fontenc}		  % Selecao de codigos de fonte.
\usepackage{graphicx}			    % Inclusão de gráficos
\usepackage[utf8]{inputenc}		% Codificacao do documento (conversão automática dos acentos)
\usepackage{txfonts}			    % Fontes virtuais
\usepackage{amsmath}
\usepackage{tikz}
\usetikzlibrary{positioning}

% ---

% --- Informações do documento ---
\title{Hands-On 12}
\author{Gustavo Lopes Rodrigues}
% ---

% ----------------- INÍCIO DO DOCUMENTO --------------------------------------
\begin{document}

    % ----------------- NOVO SLIDE --------------------------------
    \begin{frame}

    \titlepage

    \end{frame}

    %TABELA DE CONTEÚDO

    \AtBeginSection[]
    {
    \begin{frame}
      \frametitle{Conteúdo}
    \tableofcontents[currentsection]
    \end{frame}
    }

    % ----------------- NOVO SLIDE --------------------------------
    \section{Query 1}

    \begin{frame}{Query}
     
      Em SQL

      \begin{flushleft}
          SELECT A.CPF, A.Nome, B.Nome \\
          FROM Funcionarios A, Clientes B, Aluguel C, Funcionarios D \\
          WHERE A.CPF=B.CPF \\
          AND B.CPF=C.CPF\_Cliente \\
          AND B.Sexo='M' \\
          AND C.ValorPagar>50 \\
          AND A.CPF=D.CPF\_Supervisor \\
      \end{flushleft}

    \end{frame}
    % ----------------- NOVO SLIDE --------------------------------

    \begin{frame}{Query}
  
      Em Algebra Relacional
      \begin{flushleft}
        $\pi$ A.CPF, A.Nome, B.Nome $\sigma$ A.CPF = B.CPF $\wedge$ B.CPF = C.CPF\_Cliente $\wedge$ B.Sexo = 'M' $\wedge$ C.ValorPagar > 50 $\wedge$ A.CPF = D.CPF\_Supervisor ( ( ( $\rho$ A Funcionarios $\bowtie$ $\rho$ B Clientes ) $\bowtie$ $\rho$ C Aluguel ) $\bowtie$ $\rho$ D Funcionarios ) 
      \end{flushleft}
    \end{frame}
    % ----------------- NOVO SLIDE --------------------------------

    \begin{frame}{Árvore não otimizada}
     
      \begin{center}
        \begin{tikzpicture}[
          scale=0.5, transform shape,
          operation/.style={very thick, minimum size=7mm},
          table/.style={rectangle, draw=black, minimum size=5mm},
          ]
          %Nodes
          \node[operation]  (first)  at (0,0) {$\pi$ A.CPF, A.Nome, B.Nome};
          \node[operation]  (second) at (0,-2) {$\sigma$ A.CPF = B.CPF $\wedge$ C.ValorPagar > 100 $\wedge$ B.CPF = C.CPF\_Cliente $\wedge$ D.Valor < 50 $\wedge$ A.CPF\_Supervisor = null $\wedge$ A.CPF = C.CPF\_Funcionario };
          \node[operation]  (third)  at (2,-4.5) {$\times$};
          \node[table]      (tableD) at (2,-6.5) {D};
          \node[operation]  (fourth) at (-2,-4.5){$\times$};
          \node[table]      (tableC) at (0,-6.5) {C};
          \node[operation]  (fifth)  at (-4,-6.5){$\times$};
          \node[table]      (tableA) at (-6,-8.5) {A};
          \node[table]      (tableB) at (-2,-8.5) {B};
          
          %Lines
          \draw[-] (first.south) -- (second.north);
          \draw[-] (second.south) -- (third.north);
          \draw[-] (third.south) -- (tableD.north);
          \draw[-] (second.south) -- (fourth.north);
          \draw[-] (fourth.south) -- (tableC.north);
          \draw[-] (fourth.south) -- (fifth.north);
          \draw[-] (fifth.south) -- (tableA.north);
          \draw[-] (fifth.south) -- (tableB.north);
          \end{tikzpicture}
      \end{center}

  \end{frame}
  % ----------------- NOVO SLIDE --------------------------------

  \section{Query 2}

  \begin{frame}{Query}
  
    Em SQL

    \begin{flushleft}
      SELECT A.Nome, C.Nome \\
      FROM Filmes A, AtoresEmFilmes B, Atores C, Midias D \\
      WHERE A.Codigo=B.CodFilme \\
      AND B.CodAtor=C.Codigo \\
      AND A.Genero="Aventura" \\
      AND A.Codigo=D.CodFilme \\
      AND D.PrecoDiaria>10 \\
    \end{flushleft}

  \end{frame}
  % ----------------- NOVO SLIDE --------------------------------
  \begin{frame}{Query}
  
    Em Algebra Relacional
    \begin{flushleft}
      $\pi$ A.Nome, C.Nome $\sigma$ A.Codigo = B.CodFilme $\wedge$ B.CodAtor = C.Codigo $\wedge$ A.Genero = 'Aventura' $\wedge$ A.Codigo = D.CodFilme $\wedge$ D.PrecoDiaria > 10 ( ( ( $\rho$ A Filmes $\bowtie$ $\rho$ B AtoresEmFilmes ) $\bowtie$ $\rho$ C Atores ) $\bowtie$ $\rho$ D Midias ) 
    \end{flushleft}
  \end{frame}
  % ----------------- NOVO SLIDE --------------------------------

  \begin{frame}{Árvore não otimizada}
   
    \begin{center}
      \begin{tikzpicture}[
        scale=0.5, transform shape,
        operation/.style={very thick, minimum size=7mm},
        table/.style={rectangle, draw=black, minimum size=5mm},
        ]
        %Nodes
        \node[operation]  (first)  at (0,0) {$\pi$ A.CPF, A.Nome, B.Nome};
        \node[operation]  (second) at (0,-2) {$\sigma$ A.Codigo = B.CodFilme $\wedge$ B.CodAtor = C.Codigo $\wedge$ A.Genero = 'Aventura' $\wedge$ A.Codigo = D.CodFilme $\wedge$ D.PrecoDiaria > 10 };
        \node[operation]  (third)  at (2,-4.5) {$\times$};
        \node[table]      (tableD) at (2,-6.5) {D};
        \node[operation]  (fourth) at (-2,-4.5){$\times$};
        \node[table]      (tableC) at (0,-6.5) {C};
        \node[operation]  (fifth)  at (-4,-6.5){$\times$};
        \node[table]      (tableA) at (-6,-8.5) {A};
        \node[table]      (tableB) at (-2,-8.5) {B};
        
        %Lines
        \draw[-] (first.south) -- (second.north);
        \draw[-] (second.south) -- (third.north);
        \draw[-] (third.south) -- (tableD.north);
        \draw[-] (second.south) -- (fourth.north);
        \draw[-] (fourth.south) -- (tableC.north);
        \draw[-] (fourth.south) -- (fifth.north);
        \draw[-] (fifth.south) -- (tableA.north);
        \draw[-] (fifth.south) -- (tableB.north);
        \end{tikzpicture}
    \end{center}

  \end{frame}
% ----------------- NOVO SLIDE --------------------------------
  \section{Query 3}

  \begin{frame}{Query}
   
    Em SQL
    
    \begin{flushleft}
      SELECT A.CPF, A.Nome, B.Nome \\
      FROM Funcionarios A, Clientes B, Aluguel C, Pagamentos D \\
      WHERE A.CPF=B.CPF \\
      AND C.ValorPagar>100 \\
      AND B.CPF=C.CPF\_Cliente \\
      AND D.Valor<50 \\
      AND A.CPF\_Supervisor IS NULL \\
      AND A.CPF=C.CPF\_Funcionario  \\
    \end{flushleft}

  \end{frame}
   % ----------------- NOVO SLIDE --------------------------------
  \begin{frame}{Query}
  
    Em Algebra Relacional
    \begin{flushleft}
      $\pi$ A.CPF, A.Nome, B.Nome $\sigma$ A.CPF = B.CPF $\wedge$ C.ValorPagar > 100 $\wedge$ B.CPF = C.CPF\_Cliente $\wedge$ D.Valor < 50 $\wedge$ A.CPF\_Supervisor = null $\wedge$ A.CPF = C.CPF\_Funcionario ( ( ( $\rho$ A Funcionarios $\bowtie$ $\rho$ B Clientes ) $\bowtie$ $\rho$ C Aluguel ) $\bowtie$ $\rho$ D Pagamentos ) 
    \end{flushleft}
  \end{frame}
  % ----------------- NOVO SLIDE --------------------------------

  \begin{frame}{Árvore não otimizada}
   
    \begin{center}
      \begin{tikzpicture}[
        scale=0.5, transform shape,
        operation/.style={very thick, minimum size=7mm},
        table/.style={rectangle, draw=black, minimum size=5mm},
        ]
        %Nodes
        \node[operation]  (first)  at (0,0) {$\pi$ A.CPF, A.Nome, B.Nome};
        \node[operation]  (second) at (0,-2) {$\sigma$ A.CPF = B.CPF $\wedge$ C.ValorPagar > 100 $\wedge$ B.CPF = C.CPF\_Cliente $\wedge$ D.Valor < 50 $\wedge$ A.CPF\_Supervisor = null $\wedge$ A.CPF = C.CPF\_Funcionario };
        \node[operation]  (third)  at (2,-4.5) {$\sigma$A.CPF=P.CPF};
        \node[table]      (tableD) at (2,-6.5) {D};
        \node[operation]  (fourth) at (-2,-4.5){$\sigma$A.CPF=P.CPF};
        \node[table]      (tableC) at (0,-6.5) {C};
        \node[operation]  (fifth)  at (-4,-6.5){$\sigma$A.CPF=P.CPF};
        \node[table]      (tableA) at (-6,-8.5) {A};
        \node[table]      (tableB) at (-2,-8.5) {B};
        
        %Lines
        \draw[-] (first.south) -- (second.north);
        \draw[-] (second.south) -- (third.north);
        \draw[-] (third.south) -- (tableD.north);
        \draw[-] (second.south) -- (fourth.north);
        \draw[-] (fourth.south) -- (tableC.north);
        \draw[-] (fourth.south) -- (fifth.north);
        \draw[-] (fifth.south) -- (tableA.north);
        \draw[-] (fifth.south) -- (tableB.north);
        \end{tikzpicture}
    \end{center}
    
  \end{frame}
% ----------------- NOVO SLIDE --------------------------------
\begin{frame}{Plano de execução}
  
  \begin{enumerate}
    \item Teste1
    \item Teste2
    \item Teste3
  \end{enumerate}
\end{frame}
% ----------------- FIM DO DOCUMENTO -----------------------------------------
\end{document}
