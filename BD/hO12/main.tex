\documentclass[aspectratio=169]{beamer}	 	


\usetheme{Warsaw}

\setbeamercolor{normal text}{fg=white,bg=black!90}
\setbeamercolor{structure}{fg=white}

\setbeamercolor{alerted text}{fg=red!85!black}

\setbeamercolor{item projected}{use=item,fg=black,bg=item.fg!35}

\setbeamercolor*{palette primary}{use=structure,fg=structure.fg}
\setbeamercolor*{palette secondary}{use=structure,fg=structure.fg!95!white}
\setbeamercolor*{palette tertiary}{use=structure,fg=structure.fg!90!black}
\setbeamercolor*{palette quaternary}{use=structure,fg=structure.fg!95!black,bg=black!80}

\setbeamercolor*{framesubtitle}{fg=white}

\setbeamercolor*{block title}{parent=structure,bg=black!60}
\setbeamercolor*{block body}{fg=black,bg=black!10}
\setbeamercolor*{block title alerted}{parent=alerted text,bg=black!15}
\setbeamercolor*{block title example}{parent=example text,bg=black!15}
% para fontes matemáticas
% Enconte mais temas e cores em http://www.hartwork.org/beamer-theme-matrix/ 
% Veja também http://deic.uab.es/~iblanes/beamer_gallery/index.html

% Customizações de Cores: fg significa cor do texto e bg é cor do fundo

% informações do PDF
\makeatletter
\hypersetup{
     	%pagebackref=true,
		pdftitle={\@title}, 
		pdfauthor={\@author},
    	pdfsubject={Hands-On 12},
	    pdfcreator={GLR},
		pdfkeywords={HandsOn12}{LaTeX}, 
		colorlinks=true,       		% false: boxed links; true: colored links
    	linkcolor=black,          	% color of internal links
    	citecolor=blue,        		% color of links to bibliography
    	filecolor=magenta,      		% color of file links
		urlcolor=blue,
		bookmarksdepth=4
}
\makeatother

% ---
% PACOTES
% ---
\usepackage[alf]{abntex2cite}	% Citações padrão ABNT
\usepackage[brazil]{babel}		% Idioma do documento
\usepackage{color}			      % Controle das cores
\usepackage[T1]{fontenc}		  % Selecao de codigos de fonte.
\usepackage{graphicx}			    % Inclusão de gráficos
\usepackage[utf8]{inputenc}		% Codificacao do documento (conversão automática dos acentos)
\usepackage{txfonts}			    % Fontes virtuais
\usepackage{amsmath}
\usepackage{tikz}
\usetikzlibrary{positioning}

% ---

% --- Informações do documento ---
\title{Hands-On 12}
% ---

% ----------------- INÍCIO DO DOCUMENTO --------------------------------------
\begin{document}
    % ----------------- NOVO SLIDE --------------------------------
    \begin{frame}

    \titlepage

    

    \end{frame}

    %TABELA DE CONTEÚDO

    \AtBeginSection[]
    {
    \begin{frame}
      \frametitle{Conteúdo}
    \tableofcontents[currentsection]
    \end{frame}
    }

    % ----------------- NOVO SLIDE --------------------------------]
    \section{Query 1}

    \begin{frame}{Query}
     
      Em SQL

      \begin{flushleft}
          SELECT A.CPF, A.Nome, B.Nome \\
          FROM Funcionarios A, Clientes B, Aluguel C, Funcionarios D \\
          WHERE A.CPF=B.CPF \\
          AND B.CPF=C.CPF\_Cliente \\
          AND B.Sexo='M' \\
          AND C.ValorPagar>50 \\
          AND A.CPF=D.CPF\_Supervisor \\
      \end{flushleft}

    \end{frame}
    % ----------------- NOVO SLIDE --------------------------------

    \begin{frame}{Query}
  
      Em Algebra Relacional
      \begin{flushleft}
        $\pi$ A.CPF, A.Nome, B.Nome $\sigma$ A.CPF = B.CPF $\wedge$ B.CPF = C.CPF\_Cliente $\wedge$ B.Sexo = 'M' $\wedge$ C.ValorPagar > 50 $\wedge$ A.CPF = D.CPF\_Supervisor ( ( ( $\rho$ A Funcionarios $\bowtie$ $\rho$ B Clientes ) $\bowtie$ $\rho$ C Aluguel ) $\bowtie$ $\rho$ D Funcionarios ) 
      \end{flushleft}
    \end{frame}
    % ----------------- NOVO SLIDE --------------------------------
    

    \begin{frame}{Árvore não otimizada}
     
      \begin{center}
        \begin{tikzpicture}[
          scale=0.5, transform shape,
          operation/.style={very thick, minimum size=7mm},
          table/.style={rectangle, draw=black, minimum size=5mm},
          ]
          %Nodes
          \node[operation]  (first)  at (0,1) {$\pi$ A.CPF, A.Nome, B.Nome};
          \node[operation]  (second) at (0,0) {$\sigma$ A.CPF = B.CPF $\wedge$ B.CPF = C.CPF\_Cliente $\wedge$ B.Sexo = 'M' $\wedge$ C.ValorPagar > 50 $\wedge$ A.CPF = D.CPF\_Supervisor };
          \node[table]      (tableD) at (2,-2.5) {D};
          \node[operation]  (fourth) at (-2,-2.5){$\times$};
          \node[table]      (tableC) at (0,-4.5) {C};
          \node[operation]  (fifth)  at (-4,-4.5){$\times$};
          \node[table]      (tableA) at (-6,-6.5) {A};
          \node[table]      (tableB) at (-2,-6.5) {B};
          
          %Lines
          \draw[-] (first.south) -- (second.north);
          \draw[-] (second.south) -- (tableD.north);
          \draw[-] (second.south) -- (fourth.north);
          \draw[-] (fourth.south) -- (tableC.north);
          \draw[-] (fourth.south) -- (fifth.north);
          \draw[-] (fifth.south) -- (tableA.north);
          \draw[-] (fifth.south) -- (tableB.north);
          \end{tikzpicture}
      \end{center}

  \end{frame}
  % ----------------- NOVO SLIDE --------------------------------
  \begin{frame}{Árvore otimizada}
   
    \begin{center}
      \begin{tikzpicture}[
        scale=0.5, transform shape,
        operation/.style={very thick, minimum size=7mm},
        table/.style={rectangle, draw=black, minimum size=5mm},
        ]
        %Nodes
        \node[operation]  (first)  at (0,1) {$\pi$ A.CPF, A.Nome, B.Nome};
        \node[operation]  (second) at (0,0) {$\bowtie$ A.CPF=D.CPF\_SUPERVISOR};
        \node[operation]  (omega6) at (3,-2.5) {$\pi$ D.CPF\_SUPERVISOR};
        \node[table]      (tableD) at (3,-3.5) {D};
        \node[operation]  (fourth) at (-3,-2.5){$\bowtie$ B.CPF=C.CPF\_CLIENTE};
        \node[operation]  (omega1) at (0,-5.5){$\sigma$ ValorPagar > 50};
        \node[operation]  (omega5) at (0,-4.5) {$\pi$ C.CPF\_Cliente};
        \node[table]      (tableC) at (0,-6.5) {C};
        \node[operation]  (fifth)  at (-6,-4.5){$\bowtie$ A.CPF=B.CPF};
        \node[operation]  (omega3) at (-9,-6.5) {$\pi$ A.Nome, A.CPF};
        \node[table]      (tableA) at (-9,-7.5) {A};
        \node[operation]  (omega2) at (-3,-7.5) {$\sigma$ B.Sexo='M'};
        \node[operation]  (omega4) at (-3,-6.5) {$\pi$ B.Nome, B.CPF};
        \node[table]      (tableB) at (-3,-8.5) {B};
        
        %Lines
        \draw[-] (first.south) -- (second.north);
        \draw[-] (second.south) -- (omega6.north);
        \draw[-] (omega6.south) -- (tableD.north);
        \draw[-] (second.south) -- (fourth.north);
        \draw[-] (fourth.south) -- (omega5.north);
        \draw[-] (omega5.south) -- (omega1.north);
        \draw[-] (omega1.south) -- (tableC.north);
        \draw[-] (fourth.south) -- (fifth.north);
        \draw[-] (fifth.south) -- (omega3.north);
        \draw[-] (omega3.south) -- (tableA.north);
        \draw[-] (fifth.south) -- (omega4.north);
        \draw[-] (omega4.south) -- (omega2.north);
        \draw[-] (omega2.south) -- (tableB.north);
        \end{tikzpicture}
    \end{center}

  \end{frame}
  % ----------------- NOVO SLIDE --------------------------------
  \begin{frame}{Query Otimizada}
    Em Algebra Relacional
    \begin{flushleft}
     A $\rightarrow$ $\pi$ A.Nome, A.CPF $\rho$ A Funcionarios 

     B $\rightarrow$ $\pi$ B.Nome, B.CPF $\sigma$ B.Sexo ='M' $\rho$ B Clientes

     AB $\rightarrow$ A $\bowtie$ A.CPF=B.CPF B

     C $\rightarrow$ $\pi$ C.CPF\_Cliente $\sigma$ ValorPagar>50 $\rho$ C Aluguel

     CAB $\rightarrow$ AB $\bowtie$ B.CPF=C.CPF\_Cliente C

     D $\rightarrow$ $\pi$ D.CPF\_Supervisor $\rho$ D Funcionarios 

     DCAB $\rightarrow$ CAB $\bowtie$ A.CPF=D.CPF\_Supervisor D

     $\pi$ A.CPF, A.Nome, B.Nome DCAB
    \end{flushleft}
  \end{frame}
  % ----------------- NOVO SLIDE --------------------------------
  \begin{frame}{Plano de execução}
    
    \begin{enumerate}
      \item Pesquisa Linear
      \item Junção Loop Único
      \item Pesquisa Linear 
      \item Junção de Loop Único 
      \item Junção de Loop Único 
    \end{enumerate}
  \end{frame}
  % ----------------- NOVO SLIDE --------------------------------

  \section{Query 2}

  \begin{frame}{Query}
  
    Em SQL

    \begin{flushleft}
      SELECT A.Nome, C.Nome \\
      FROM Filmes A, AtoresEmFilmes B, Atores C, Midias D \\
      WHERE A.Codigo=B.CodFilme \\
      AND B.CodAtor=C.Codigo \\
      AND A.Genero="Aventura" \\
      AND A.Codigo=D.CodFilme \\
      AND D.PrecoDiaria>10 \\
    \end{flushleft}

  \end{frame}
  % ----------------- NOVO SLIDE --------------------------------
  \begin{frame}{Query}
  
    Em Algebra Relacional
    \begin{flushleft}
      $\pi$ A.Nome, C.Nome $\sigma$ A.Codigo = B.CodFilme $\wedge$ B.CodAtor = C.Codigo $\wedge$ A.Genero = 'Aventura' $\wedge$ A.Codigo = D.CodFilme $\wedge$ D.PrecoDiaria > 10 ( ( ( $\rho$ A Filmes $\bowtie$ $\rho$ B AtoresEmFilmes ) $\bowtie$ $\rho$ C Atores ) $\bowtie$ $\rho$ D Midias ) 
    \end{flushleft}
  \end{frame}
  % ----------------- NOVO SLIDE --------------------------------

  \begin{frame}{Árvore não otimizada}
   
    \begin{center}
      \begin{tikzpicture}[
        scale=0.5, transform shape,
        operation/.style={very thick, minimum size=7mm},
        table/.style={rectangle, draw=black, minimum size=5mm},
        ]
        %Nodes
        \node[operation]  (first)  at (0,0) {$\pi$ A.CPF, A.Nome, C.Nome};
        \node[operation]  (second) at (0,-1) {$\sigma$ A.Codigo = B.CodFilme $\wedge$ B.CodAtor = C.Codigo $\wedge$ A.Genero = 'Aventura' $\wedge$ A.Codigo = D.CodFilme $\wedge$ D.PrecoDiaria > 10 };
        \node[table]      (tableD) at (2,-3.5) {D};
        \node[operation]  (fourth) at (-2,-3.5){$\times$};
        \node[table]      (tableC) at (0,-5.5) {C};
        \node[operation]  (fifth)  at (-4,-5.5){$\times$};
        \node[table]      (tableA) at (-6,-7.5) {A};
        \node[table]      (tableB) at (-2,-7.5) {B};
        
        %Lines
        \draw[-] (first.south) -- (second.north);
        \draw[-] (second.south) -- (tableD.north);
        \draw[-] (second.south) -- (fourth.north);
        \draw[-] (fourth.south) -- (tableC.north);
        \draw[-] (fourth.south) -- (fifth.north);
        \draw[-] (fifth.south) -- (tableA.north);
        \draw[-] (fifth.south) -- (tableB.north);
        \end{tikzpicture}
    \end{center}

  \end{frame}
  % ----------------- NOVO SLIDE --------------------------------
  \begin{frame}{Árvore otimizada}
   
    \begin{center}
      \begin{tikzpicture}[
        scale=0.5, transform shape,
        operation/.style={very thick, minimum size=7mm},
        table/.style={rectangle, draw=black, minimum size=5mm},
        ]
        %Nodes
        \node[operation]  (first)  at (0,1)     {$\pi$ A.CPF, A.Nome, C.Nome};
        \node[operation]  (second) at (0,0)     {$\bowtie$ A.Codigo=D.CodFilme};
        \node[operation]  (omega1) at (3,-2.5)  {$\pi$ D.CodFilme};
        \node[operation]  (omega3) at (3,-3.5)  {$\sigma$ D.PrecoDiaria > 10};
        \node[table]      (tableD) at (3,-4.5)  {D};
        \node[operation]  (fourth) at (-3,-2.5) {$\bowtie$ B.CodAtor=C.Codigo};
        \node[operation]  (omega4) at (0,-4.5)  {$\pi$ C.Nome, C.Codigo};
        \node[table]      (tableC) at (0,-5.5)  {C};
        \node[operation]  (fifth)  at (-6,-4.5) {$\bowtie$ A.Codigo=B.CodFilme};
        \node[operation]  (omega2) at (-9,-7.5) {$\sigma$ A.Genero='Aventura'};
        \node[operation]  (omega5) at (-9,-6.5) {$\pi$ A.Nome, A.Codigo};
        \node[table]      (tableA) at (-9,-8.5) {A};
        \node[table]      (tableB) at (-3,-6.5) {B};
        
        %Lines
        \draw[-] (first.south) -- (second.north);
        \draw[-] (second.south) -- (omega1.north);
        \draw[-] (omega1.south) -- (omega3.north);
        \draw[-] (omega3.south) -- (tableD.north);
        \draw[-] (second.south) -- (fourth.north);
        \draw[-] (fourth.south) -- (omega4.north);
        \draw[-] (omega4.south) -- (tableC.north);
        \draw[-] (fourth.south) -- (fifth.north);
        \draw[-] (fifth.south) -- (omega5.north);
        \draw[-] (omega5.south) -- (omega2.north);
        \draw[-] (omega2.south) -- (tableA.north);
        \draw[-] (fifth.south) -- (tableB.north);
        \end{tikzpicture}
    \end{center}

  \end{frame}
  % ----------------- NOVO SLIDE --------------------------------
  \begin{frame}{Query Otimizada}
  
    Em Algebra Relacional
    \begin{flushleft}
      A $\rightarrow$ $\pi$ A.Nome, A.Codigo $\sigma$ A.Genero='Aventura' $\rho$ A Midias 

      B $\rightarrow$ $\rho$ B AtoresEmFilmes
 
      AB $\rightarrow$ A $\bowtie$ A.Codigo=B.Codigo B
 
      C $\rightarrow$ $\pi$ C.Nome, C.Codigo $\rho$ C Atores
 
      CAB $\rightarrow$ AB $\bowtie$ B.CodAtor=C.Codigo C
 
      D $\rightarrow$ $\pi$ D.CodFilme $\sigma$ D.PrecoDiaria>10 $\rho$ D Midias 
 
      DCAB $\rightarrow$ CAB $\bowtie$ A.Codigo=D.CodFilme D
 
      $\pi$ A.CPF, A.Nome, C.Nome DCAB
    \end{flushleft}
  \end{frame}
% ----------------- NOVO SLIDE --------------------------------
  \begin{frame}{Plano de execução}
    
    \begin{enumerate}
      \item Pesquisa Linear 
      \item Junção de Loop Único 
      \item Junção de Loop Único 
      \item Pesquisa Linear 
      \item Junção de Loop Único
    \end{enumerate}
  \end{frame}
  % ----------------- NOVO SLIDE --------------------------------
  \section{Query 3}

  \begin{frame}{Query}
    
    Em SQL
    
    \begin{flushleft}
      SELECT A.CPF, A.Nome, B.Nome \\
      FROM Funcionarios A, Clientes B, Aluguel C, Pagamentos D \\
      WHERE A.CPF=B.CPF \\
      AND C.ValorPagar>100 \\
      AND B.CPF=C.CPF\_Cliente \\
      AND D.Valor<50 \\
      AND A.CPF\_Supervisor IS NULL \\
      AND A.CPF=C.CPF\_Funcionario  \\
    \end{flushleft}

  \end{frame}
    % ----------------- NOVO SLIDE --------------------------------
  \begin{frame}{Query}

    Em Algebra Relacional
    \begin{flushleft}
      $\pi$ A.CPF, A.Nome, B.Nome $\sigma$ A.CPF = B.CPF $\wedge$ C.ValorPagar > 100 $\wedge$ B.CPF = C.CPF\_Cliente $\wedge$ D.Valor < 50 $\wedge$ A.CPF\_Supervisor = null $\wedge$ A.CPF = C.CPF\_Funcionario ( ( ( $\rho$ A Funcionarios $\bowtie$ $\rho$ B Clientes ) $\bowtie$ $\rho$ C Aluguel ) $\bowtie$ $\rho$ D Pagamentos ) 
    \end{flushleft}
  \end{frame}
    % ----------------- NOVO SLIDE --------------------------------
  \begin{frame}{Árvore não otimizada}
    
    \begin{center}
      \begin{tikzpicture}[
        scale=0.5, transform shape,
        operation/.style={very thick, minimum size=7mm},
        table/.style={rectangle, draw=black, minimum size=5mm},
        ]
        %Nodes
        \node[operation]  (first)  at (0,1) {$\pi$ A.CPF, A.Nome, B.Nome};
        \node[operation]  (second) at (0,0) {$\sigma$ A.CPF = B.CPF $\wedge$ C.ValorPagar > 100 $\wedge$ B.CPF = C.CPF\_Cliente $\wedge$ D.Valor < 50 $\wedge$ A.CPF\_Supervisor = null $\wedge$ A.CPF = C.CPF\_Funcionario };
        \node[table]      (tableD) at (2,-2.5) {D};
        \node[operation]  (fourth) at (-2,-2.5){$\times$};
        \node[table]      (tableC) at (0,-4.5) {C};
        \node[operation]  (fifth)  at (-4,-4.5){$\times$};
        \node[table]      (tableA) at (-6,-6.5) {A};
        \node[table]      (tableB) at (-2,-6.5) {B};
        
        %Lines
        \draw[-] (first.south) -- (second.north);
        \draw[-] (second.south) -- (tableD.north);
        \draw[-] (second.south) -- (fourth.north);
        \draw[-] (fourth.south) -- (tableC.north);
        \draw[-] (fourth.south) -- (fifth.north);
        \draw[-] (fifth.south) -- (tableA.north);
        \draw[-] (fifth.south) -- (tableB.north);
        \end{tikzpicture}
    \end{center}
    
  \end{frame}
  % ----------------- NOVO SLIDE --------------------------------
  \begin{frame}{Árvore otimizada}
  
    \begin{center}
      \begin{tikzpicture}[
        scale=0.5, transform shape,
        operation/.style={very thick, minimum size=7mm},
        table/.style={rectangle, draw=black, minimum size=5mm},
        ]
        %Nodes
        \node[operation]  (first)  at (0,0)     {$\pi$ A.CPF, A.Nome, B.Nome};
        \node[operation]  (second) at (0,-1)    {$\bowtie$ A.CPF = D.CPF\_Cliente};
        \node[operation]  (omega6) at (3,-3.5)  {$\pi$ D.CPF\_Cliente };
        \node[operation]  (omega1) at (3,-4.5)  {$\sigma$ D.Valor<50 };
        \node[table]      (tableD) at (3,-5.5)  {D};
        \node[operation]  (fourth) at (-3,-3.5) {$\bowtie$ B.CPF=C.CPF\_Cliente};
        \node[operation]  (omega2) at (0,-6.5)  {$\sigma$ C.Valor>100 };
        \node[operation]  (omega5) at (0,-5.5)  {$\pi$ C.CPF\_Cliente};
        \node[table]      (tableC) at (0,-7.5)  {C};
        \node[operation]  (fifth)  at (-6,-5.5) {$\bowtie$ A.CPF=B.CPF};
        \node[operation]  (omega4) at (-9,-7.5){$\pi$ A.CPF, A.Nome};
        \node[operation]  (omega3) at (-9,-8.5){$\sigma$ A.CPF\_Supervisor = null};
        \node[table]      (tableA) at (-9,-9.5) {A};
        \node[operation]  (omega7) at (-3,-7.5) {$\pi$ B.CPF, B.Nome};
        \node[table]      (tableB) at (-3,-8.5) {B};
        
        %Lines
        \draw[-] (first.south) -- (second.north);
        \draw[-] (second.south) -- (omega6.north);
        \draw[-] (omega6.south) -- (omega1.north);
        \draw[-] (omega1.south) -- (tableD.north);
        \draw[-] (second.south) -- (fourth.north);
        \draw[-] (fourth.south) -- (omega5.north);
        \draw[-] (omega5.south) -- (omega2.north);
        \draw[-] (omega2.south) -- (tableC.north);
        \draw[-] (fourth.south) -- (fifth.north);
        \draw[-] (fifth.south) -- (omega4.north);
        \draw[-] (omega4.south) -- (omega3.north);
        \draw[-] (omega3.south) -- (tableA.north);
        \draw[-] (fifth.south) -- (omega7.north);
        \draw[-] (omega7.south) -- (tableB.north);
        \end{tikzpicture}
    \end{center}
    
  \end{frame}
  % ----------------- NOVO SLIDE --------------------------------
  \begin{frame}{Query Otimizada}

    Em Algebra Relacional
    \begin{flushleft}
      A $\rightarrow$ $\pi$ A.Nome, A.CPF $\sigma$ A.CPF\_Supervisor=null $\rho$ A Funcionarios 

      B $\rightarrow$ $\pi$ B.Nome, B.CPF $\rho$ B Clientes
 
      AB $\rightarrow$ A $\bowtie$ A.CPF=B.CPF B
 
      C $\rightarrow$ $\pi$ C.CPF\_Cliente $\sigma$ ValorPagar>100 $\rho$ C Aluguel
 
      CAB $\rightarrow$ AB $\bowtie$ B.CPF=C.CPF\_Cliente C
 
      D $\rightarrow$ $\pi$ D.CPF\_Cliente $\sigma$ ValorPagar<50 $\rho$ D Funcionarios 
 
      DCAB $\rightarrow$ CAB $\bowtie$ A.CPF=D.CPF\_Supervisor D
 
      $\pi$ A.CPF, A.Nome, B.Nome DCAB
    \end{flushleft}
  \end{frame}
  % ----------------- NOVO SLIDE --------------------------------
  \begin{frame}{Plano de execução}
    
    \begin{enumerate}
      \item Pesquisa Linear 
      \item Junção de Loop Único
      \item Pesquisa Linear 
      \item Junção de Loop Único 
      \item Pesquisa Linear 
      \item Junção de Loop Único
    \end{enumerate}
  \end{frame}
% ----------------- FIM DO DOCUMENTO -----------------------------------------
\end{document}
