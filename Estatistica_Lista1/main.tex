\documentclass[12pt]{article}
\usepackage[T1]{fontenc}
\usepackage[utf8]{inputenc}
\usepackage[brazil]{babel}
\usepackage{hyperref}
\usepackage[a4paper,top=3.5cm,left=3cm,right=3cm,bottom=2.5cm]{geometry}
\usepackage{pgfplots}
\pgfplotsset{compat=1.8}
\usepgfplotslibrary{statistics}

\newcounter{instn}
\setcounter{instn}{1}
\newcommand{\instnum}{\arabic{instn}}

\newcommand{\myline}[1]{
    \emph{\textbf{#1)}}
    \addtocounter{instn}{1}
}

%configurando identação e separação de parágrafos
\parindent 1.27cm
\parskip   6pt

\setlength\parindent{0pt}

%configurando os hyperlinks
\hypersetup{
    colorlinks=true,
    linkcolor=green,
    filecolor=magenta,      
    urlcolor=blue,
}

%títulos,autor e data
\title{\textbf{Lista 1 - Estatística}}
\author{Gustavo Lopes Rodrigues}
\date{2021}

\begin{document}
    
    \maketitle

    \section*{Fórmulas usadas}

    Media aritmética \[ M_a = \frac{x_1 + x_2 + x_3 + .... x_n }{n} \]

    Variância \[ V = \frac{\sum_{i=1}^{n} (x_i - M_a)^2}{n-1}\]

    Desvio Padrão \[ D_p = \sqrt{V}\]

    Coeficiente de Variação \[ C_v = \frac{D_p}{M_a} * 100\]

    \newpage

    \section*{Respostas}

    \myline{\instnum} \\ Media aritmética = \textbf{77.48}, Desvio Padrão = \textbf{8.17}, Coeficiente de variação \textbf{10.55}


    Analisando os dados, percebe-se que a média aritmética é de 77.48, uma margem de erro de 8.17\% além 
    de um coeficiente de variação igual a 10.55\%. Tais observações sugere que 
    as cotações diárias das ações dessa empresa neste intervalo são homogêneas, pois não exite 
    uma diferenciação muito grande entre os dados.

    \myline{\instnum} \\ Media aritmética = \textbf{56.6}, Mediana \textbf{57.5}, Desvio Padrão = \textbf{5.66}


    Examinando a pontuação de cada trabalhador, percebe-se que a media aritmética das notas é igual a 
    56.6, a mediana é 57.5, e existe uma margem de erro entre os dados de 5.56. Os dados são homogêneos,
    já que a variação entre os números não é grande(demonstrado pelo desvio padrão). Além disso, percebe-se que quatro 
    trabalhadores estão com notas abaixo da média, enquanto que o restante mantém acima, com destaque a duas notas. que superaram
    o desvio padrão. 


    \myline{\instnum} (há fazer) 


    \myline{\instnum} \\ Observando os resultados: o objetivo da indústria foi alcançada, pois eles conseguiram aumentar
    a média de 59.6\% para 75.2\%, um aumento de 26\%, mantendo um grau de desvio baixo, 9\%, indicando que os dados
    ainda possuem homogeneidade baixa.


    \myline{\instnum} \\ a) Media aritmética = \textbf{816.6}, Mediana \textbf{800}, Desvio Padrão = \textbf{212.48}
    \\ b) O mês com o valor mais alto está com um pouco menos de 2 desvios padrões.


    \myline{\instnum} \\ a) Mediana = \textbf{59.61} \\ b) Q3 = \textbf{69.18}

    c) Box Plot: 
    \begin{center}
        \begin{tikzpicture}
            \begin{axis}
                axis x line=left,
                x axis line style={opacity=0},
                xtick=\empty,
              \addplot+[
              boxplot prepared={
                median=59.61,
                upper quartile=69.18,
                lower quartile=50.87,
                upper whisker=85.18,
                lower whisker=35.46
              },
              ] coordinates {};
            \end{axis}
        \end{tikzpicture}
    \end{center}



    \myline{\instnum}

    a) Por possuir um desvio padrão baixo(14.5\%), e não ter nenhum valor muito alto,
    qualquer uma das medidas podem representar os dados

    \begin{center}
        \begin{tabular}{|c | c |} 
        \hline
        Media aritmética & 54.95 \\ 
        \hline
        Mediana & 56 \\ 
        \hline
        Moda & 40, 54, 60, 62 multimodal \\ 
        \hline
        Desvio Padrão & 14.09\% \\ 
        \hline
        Primeiro quartil & 45.0 \\ 
        \hline
        Terceiro quartil & 64.28 \\ 
        \hline
       \end{tabular}
    \end{center}

    \begin{itemize}
        \item Um pouco mais da metade (\emph{55\%}) está com a nota acima da média 
        \item Tem pelo menos 5 alunos que estão com notas preocupantes, pois estão muito abaixo da média 
        \item Tem 2 alunos que se destacaram, um que teve um destaque ainda maior.
    \end{itemize}

    b) A melhor medida pode ser o quartil, ou a mediana, já que a maior moda é igual a 0,
    e sem contar que todas as modas são muito dispersas quando comparado ao maior número de faltas (45)

    \begin{center}
        \begin{tabular}{|c | c |} 
        \hline
        Media aritmética & 4.25 \\ 
        \hline
        Mediana & 1.5 \\ 
        \hline
        Moda & 0, 1, 2, 3, 5 multimodal \\ 
        \hline
        Desvio Padrão & 9.86\% \\ 
        \hline
        Primeiro quartil & 0.25 \\ 
        \hline
        Terceiro quartil & 4.71 \\ 
        \hline
       \end{tabular}
    \end{center}

    Análse: Percebe-se que em grande parte o número de faltas entre os trabalhadores são bem próximos.
    A única exceção é o maior número, que indica a presença de um "outlier", um número muito fora do padrão.

    c) Neste caso, a melhor medida central pode ser a mediana ou a moda, pois muitas 
    pessoas não leêm o jornal. 

    \begin{center}
        \begin{tabular}{|c | c |} 
        \hline
        Media aritmética & 3.1 \\ 
        \hline
        Mediana & 0 \\ 
        \hline
        Moda & 0, 1, 11, 12 multimodal \\ 
        \hline
        Desvio Padrão & 5.06\% \\ 
        \hline
        Primeiro quartil & 0 \\ 
        \hline
        Terceiro quartil & 7.14 \\ 
        \hline
       \end{tabular}
    \end{center}

    Examinando o número de exemplares, percebe-se que este jornal não é bem vendido, 
    já que a mediana, e o valor com maior frequência nos dados é 0.

    \newpage

    d) Por se tratar de analisar pessoas com condição médica, o uso da Moda parece ser
    apropriado, assim também como a média, por conta dos dados terem uma dispersão baixa

    \begin{center}
        \begin{tabular}{|c | c |} 
        \hline
        Media aritmética & 1.65 \\ 
        \hline
        Mediana & 1.615 \\ 
        \hline
        Moda & 1.52, 1.55, 1.6, 1.63, 1.65 multimodal \\ 
        \hline
        Desvio Padrão & 0.2\% \\ 
        \hline
        Primeiro quartil & 1.56 \\ 
        \hline
        Terceiro quartil & 1.68 \\ 
        \hline
       \end{tabular}
    \end{center}

    Ao fazer o estudo, percebe-se que esse grupo de dados é multimodal(possui várias)
    modas, mais específico, as alturas: 1.52, 1.55, 1.6, 1.63, 1.65, foram as que mais apareceram.
    A média encontrada foi de 1.65

    Ainda como observação a mais, a altura 2.50 foi registrada, o que pode significar ser um possível
    "outlier", um possível número falso, pois está muito acima da dispersão.


    \myline{\instnum}

    \begin{center}
        \begin{tabular}{|c | c | c | c|} 
        \hline
        Valor faturado em milhões de Reais & $x_i$ & f & f\% \\ [3pt]
        \hline
        0,2|---0,6 & 6 & 87837 & 13,33\% \\ 
        \hline
        2 & 7 & 78 & 21,67\% \\
        \hline
        3 & 545 & 778 & 26,67\% \\
        \hline
        4 & 545 & 18744 & 20,00\% \\
        \hline
        5 & 88 & 788 & 10,00\% \\ 
        \hline
        5 & 88 & 788 & 8,33\% \\ 
        \hline
        \textbf{Total} & 88 & 788 & 100\% \\ [3pt]
        \hline
       \end{tabular}
    \end{center}

    \myline{\instnum} \\ Media aritmética = \textbf{82.75}, Mediana \textbf{82}, Desvio Padrão = \textbf{11.84}


    \myline{\instnum}


    \myline{\instnum} \\ a) Q1 = \textbf{2.30}, Q2 = \textbf{4}, Q3 = \textbf{5.53}
    \\ b) 15, o maior valor é um "outlier", pois para ser considerado um é
    preciso encontrar a faixa interquartil(FIQ), que é a difença entre o terceiro e 
    primeiro quartil. A partir dela, podemos observar o intervalo entre:
    
    \begin{equation}
        Q1 - 1.5 FIQ \rightarrow 2.30 + 1.5 (3.23) = 7.14
    \end{equation}

    \begin{equation}
        Q3 + 1.5 FIQ \rightarrow 5.53 + 1.5 (3.23) = 10.37
    \end{equation}

    O número 15 como pode ser observado, é maior do que este intervalo. 



\end{document}
