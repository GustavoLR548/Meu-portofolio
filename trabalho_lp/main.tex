\documentclass[12pt]{article}
\usepackage[T1]{fontenc}
\usepackage[utf8]{inputenc}
\usepackage[brazil]{babel}
\usepackage{hyperref}
\usepackage{amsmath}
\usepackage[a4paper,top=3.5cm,left=3cm,right=3cm,bottom=2.5cm]{geometry}

%configurando identação e separação de parágrafos
\parindent 1.27cm
\parskip   6pt

%configurando os hyperlinks
\hypersetup{
    colorlinks=true,
    linkcolor=green,
    filecolor=magenta,      
    urlcolor=blue,
}

%títulos,autor e data
\title{\textbf{Um resumo para o trabalho de LP}}
\author{Gustavo Lopes Rodrigues}
\date{Abril de 2021}

\begin{document}
    
    \maketitle

    \begin{abstract}
        O Artigo \emph{On Understanding Types,Data Abstraction, and Polymorphism}, busca fazer uma compreensão
        dos termos: tipagem, abstração de dados e polimorfismo, que junto aos conhecimentos da teoria da tipagem,
        busca criar um modelo para criação de linguagens de programação forte, com alto polimorfismo, utilizando o
        cálculo $\lambda$. O nome desse módelo é \emph{FUN}.
    \end{abstract}

    \section{Secção 1}
    \subsection{Sem tipo e com tipo}

        Para começar o artigo, o autor em vez de ir pela definição diretamente, ele 
        deu exemplos de linguagens sem tipagem e o processo de tipagem foi naturalmente
        sendo incorporado.

        Não tipagem significa ter uma tipagem só. Na memória do computador isso é representando por 
        bit strings. Em LISP, são as \emph{S-expressions}. No Cálculo lambda, são as expressões lambda. Por fim
        também tem os Sets em Set Theory. 

        A ideia de possuir tipos foi naturalmente implementada, com a necessidade de possuir diferentes 
        formatos com diferentes usos e diferentes comportamentos. Porém, ainda é muito díficil fazer uma distinção completa 
        entre a organização de sem tipos e de fazer realmente uma linguagem com tipagem. Um exemplo disso seria ter uma função 
        lambda que retorna booleano ou integer.

    \subsection{Tipagem forte e fraca}

        Uma maneira descrever tipos é comparando com uma armadura. Armaduras protegem o usuário de danos
        exteriores. Em tipagem, a armadura protege os dados de serem usados de forma não desejadas ou não
        intencionadas. Isso acontece, pois como objetos possuem certas tipagem, ele precisa seguir as 
        funcionalidade da tal tipagem. O problema é que mesmo assim, durante compilação, pode acontecer de
        essa regra não ser obdecida e causar muitos problemas. Uma solução para esse problema é a tipagem estática. 
        
        Tipagem estática significa que variáveis de um programa são definidas explicitamentes e então 
        checados durante tempo de compilação. Isso é importante para a checagem de erros, melhoria em performance , garante 
        uma estrutura a ser respeitada e de fácil leitura. Porém, existe uma outra tipo de tipagem, que garante maior 
        flexibilidade, a tipagem forte. 

        Tipagem forte são linguagens onde cada tipo de dado, são predefinidos como parte da linguagem




\end{document}