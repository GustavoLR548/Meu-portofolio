\documentclass[12pt]{article}
\usepackage[T1]{fontenc}
\usepackage[utf8]{inputenc}
\usepackage[brazil]{babel}
\usepackage{hyperref}
\usepackage[a4paper,top=3.5cm,left=3cm,right=3cm,bottom=2.5cm]{geometry}
\usepackage{pgfplots}
\usepackage{enumitem}
\usepackage{amsmath}
\usepackage{graphicx}
\pgfplotsset{compat=1.8}
\usepgfplotslibrary{statistics}

\newcounter{instn}
\setcounter{instn}{1}
\newcommand{\instnum}{\arabic{instn}}

\newcommand{\myline}[1]{
    \emph{\textbf{#1)}}
    \addtocounter{instn}{1}
}

\newenvironment{question}
 {
    \myline{\instnum} 
    }
    {
 }

%configurando identação e separação de parágrafos
\parindent 1.27cm
\parskip   6pt

\setlength\parindent{0pt}

%configurando os hyperlinks
\hypersetup{
    colorlinks=true,
    linkcolor=green,
    filecolor=magenta,      
    urlcolor=blue,
}

%títulos,autor e data
\title{\textbf{Lista 3 - Estatística}}
\author{Gustavo Lopes Rodrigues}
\date{2021}

\begin{document}
    
    \maketitle

    \section*{Respostas}

    %------Questao1--------------------------------------------------------------------------------%

    \begin{question}
       
        \begin{enumerate}[label={\textbf{\alph*)}}]

            \item 
            
            Amostragem é uma técnica de seleção de uma amostra ou um subconjunto de elementos em uma determinada
            população que possibilita o estudo das características de uma população.
            \item 
            
            Amostragem Aleatória é uma amostragem pelo qual cada elemento de tamanho igual da
            população tem a mesma chance de ser selecionado no estudo que vai ser realizada.
            \item 
            
            Amostra aleatória simples: Todas as amostras de mesmo tamanho são igualmente prováveis.
            Amostra sistemática: Combina um processo aleatório com um processo sistémico. Percorre toda população.
            \item 
            
            Cidades urbanas e Interior e estudantes estudando no setor pública e estudante de escolas privadas
            \item 
            
            50 casa distribuido em 5 ruas, sorteando 2 segunda casa em cada rua
            30 estudante de faculdade de relações internacionais do quinto período distribuído em 3 salas, sorteando as duas
            primeiro salas

        \end{enumerate}
    \end{question}

    %------Questao2--------------------------------------------------------------------------------%

    \begin{question}
        
        \begin{enumerate}[label={\textbf{\alph*)}}]

            \item e = $\frac{Z\_a}{2} * \frac{\sigma}{\sqrt{n}}$ 
            
                  e = 2.0129 * $\frac{0.22}{\sqrt{46}}$

                  e = 0.0652
            \item  
                IC(1-$\alpha$)\%= $\bar{x}$ $\pm$ Z*$\frac{\alpha}{2}$ * $\frac{\sigma}{\sqrt{N}}$

                IC(95)\% = 35.2 $\pm$ 2.0129 * $\frac{0.22}{\sqrt{46}}$

                [35.137 ; 35.2652]

        \end{enumerate}
    \end{question}

    %------Questao3--------------------------------------------------------------------------------%

    \begin{question}
        
        \begin{enumerate}[label={\textbf{\alph*)}}]

            \item IC(1-$\alpha$)\%= $\bar{x}$ $\pm$ $\tau_\frac{\alpha}{2}$ ;* $\frac{S}{\sqrt{n}}$
                  
                  IC(95)\% = 450.95 $\pm$ 2.2622 * $\frac{6.36}{\sqrt{10}}$

                  IC(95)\% = 450.95 $\pm$ 4.55

                  IC(95)\% = [446.4 ; 455.50]
        \end{enumerate}
    \end{question}

    %------Questao4--------------------------------------------------------------------------------%

    \begin{question}

        $\frac{850}{2500}$ = 0.34 $\rightarrow$ 34\%

        IC(1-$\alpha$)\%= $\hat{P}$ $\pm$ $Z_\frac{\alpha}{2}$ * $\frac{\hat{P}*(1-\hat{P})}{n}$

        IC(95)\% = 0.34 $\pm$ 1.960 * $\sqrt{\frac{0.34*(0.66)}{2500}}$ 

        IC(95)\% = 0.34 $\pm$ 0.018 

        IC(95)\% = [0.322 ; 0.358]
    \end{question}

    %------Questao5--------------------------------------------------------------------------------%

    \begin{question}
        
        n = $\frac{\hat{P}*(1-\hat{P})}{(\frac{E}{Z_\frac{\alpha}{2}})^2}$

        n = $\frac{0.6(0.4)}{(\frac{0.02}{1.96})^2}$

        n = $\frac{0.24}{0.000104123}$

        n = 2304.966241849 $\approx$ 2305 

    \end{question}

    %------Questao6--------------------------------------------------------------------------------%

    \begin{question}
        
        \begin{enumerate}[label={\textbf{\alph*)}}]

            \item n = $\frac{\hat{P}*(1-\hat{P})}{(\frac{E}{Z_\frac{\alpha}{2}})^2}$
            
                  n = $\frac{0.7(0.3)}{(\frac{0.05}{1.645})^2}$

                  n = $\frac{0.21}{0.000923864}$

                  n = 227.306183594 $\approx$ 228

                  n = $\frac{n}{1+ \frac{n}{N}}$

                  n = $\frac{228}{1+\frac{228}{100}}$

                  n = $\frac{22800}{328}$

                  n = 69.512195122 $\approx$ 70

            \item n = $(Z_\frac{\alpha}{2} * \frac{\sigma}{E})^2$

                  n = $(1.96 * \frac{3}{\sqrt{1}})^2$

                  n = $(5.88)^2$

                  n = 34.5744 $\approx$ 35

            \item n = $\frac{\hat{P}*(1-\hat{P})}{(\frac{E}{Z_\frac{\alpha}{2}})^2}$
            
                  n = $\frac{0.5(0.5)}{(\frac{0.01}{2.576})^2}$

                  n = $\frac{0.25}{0.00001507}$

                  n = 16589,25016589 $\approx$ 16590

                  n = $\frac{n}{1+ \frac{n}{N}}$

                  n = $\frac{16590}{1+\frac{16590}{500}}$

                  n = $\frac{8295000}{17090}$

                  n = 485,371562317 $\approx$ 486

        \end{enumerate}
    \end{question}

    %------Questao7--------------------------------------------------------------------------------%

    \begin{question}
        
        \begin{enumerate}[label={\textbf{\alph*)}}]

            \item IC(1-$\alpha$)\%= $\bar{x}$ $\pm$ $\tau_\frac{\alpha}{2}$ ;* $\frac{S}{\sqrt{n}}$
                  
                  IC(99)\% = 800 $\pm$ 2.576 * $\frac{100}{\sqrt{400}}$

                  IC(95)\% = 800 $\pm$ 12.88

                  IC(95)\% = [787.12 ; 812.88]

            \item n = $(Z_\frac{\alpha}{2} * \frac{\sigma}{E})^2$
            
                  n = $(1.96 * \frac{100}{7.84})^2$

                  n = $(24.99)^2$

                  n = 624.5
        \end{enumerate}
    \end{question}

    %------Questao8--------------------------------------------------------------------------------%

    \begin{question}

        IC(1-$\alpha$)\%= $\hat{P}$ $\pm$ $Z_\frac{\alpha}{2}$ * $\frac{\hat{P}*(1-\hat{P})}{n}$

        IC(95)\% = 0.7 $\pm$ 1.645 * $\sqrt{\frac{0.7*(0.3)}{625}}$ 

        IC(95)\% = 0.7 $\pm$ 0.03 

        IC(95)\% = [0.67 ; 0.73]
        
    \end{question}

    %------Questao9--------------------------------------------------------------------------------%

    \begin{question}
        
        \begin{enumerate}[label={\textbf{\alph*)}}]

            \item 
                  n = $\frac{\hat{P}*(1-\hat{P})}{(\frac{E}{Z_\frac{\alpha}{2}})^2}$
            
                  n = $\frac{0.6(0.4)}{(\frac{0.01}{1.282})^2}$

                  n = $\frac{0.24}{0.000060845}$

                  n = 3944,449009779 $\approx$ 3945

                  n = $\frac{n}{1+ \frac{n}{N}}$

                  n = $\frac{3945}{1+\frac{3945}{100}}$

                  n = $\frac{394500}{4045}$

                  n = 97,527812114 $\approx$ 98
            
            \item 

                 IC(1-$\alpha$)\%= $\hat{P}$ $\pm$ $Z_\frac{\alpha}{2}$ * $\frac{\hat{P}*(1-\hat{P})}{n}$

                 IC(95)\% = 0.55 $\pm$ 1.96 * $\sqrt{\frac{0.55*(0.45)}{100}}$ 
        
                 IC(95)\% = 0.55 $\pm$ 0,097508769
        
                 IC(95)\% = [0,4524 ; 0,6475]
        \end{enumerate}
    \end{question}

    %------Questao10--------------------------------------------------------------------------------%

    \begin{question}
        
        \begin{enumerate}[label={\textbf{\alph*)}}]

            \item 
            
                 IC(1-$\alpha$)\%= $\hat{P}$ $\pm$ $Z_\frac{\alpha}{2}$ * $\frac{\hat{P}*(1-\hat{P})}{n}$

                 IC(95)\% = 0.33 $\pm$ 1.96 * $\sqrt{\frac{0.33*(0.67)}{300}}$ 
    
                 IC(95)\% = 0.33 $\pm$ 0,053
    
                 IC(95)\% = [0,277 ; 0,383]

            \item 
                
                 n = $\frac{\hat{P}*(1-\hat{P})}{(\frac{E}{Z_\frac{\alpha}{2}})^2}$
                
                 n = $\frac{0.33(0.67)}{(\frac{0.02}{1.96})^2}$

                 n = $\frac{0.2211}{0.000104123}$

                 n = 2123,450150303 $\approx$ 2124

                 n = $\frac{n}{1+ \frac{n}{N}}$

                 n = $\frac{2124}{1+\frac{2124}{300}}$

                 n = $\frac{637200}{2424}$

                 n = 262,87128712 $\approx$ 263

        \end{enumerate}
    \end{question}

    %------Questao11--------------------------------------------------------------------------------%

    \begin{question}
        
        \begin{enumerate}[label={\textbf{\alph*)}}]

            \item 
            
            \item 
       
        \end{enumerate}
    \end{question}

    %------Questao12--------------------------------------------------------------------------------%

    \begin{question}
        
        \begin{enumerate}[label={\textbf{\alph*)}}]
            \item 
            \item 
        \end{enumerate}
    \end{question}

    %------Questao13--------------------------------------------------------------------------------%

    \begin{question}
        
        \begin{enumerate}[label={\textbf{\alph*)}}]
            \item 
            \item 
        \end{enumerate}
    \end{question}

    %------Questao14--------------------------------------------------------------------------------%

    \begin{question}
        
        \begin{enumerate}[label={\textbf{\alph*)}}]
            \item 
            \item 
        \end{enumerate}
    \end{question}

    %------Questao15--------------------------------------------------------------------------------%

    \begin{question}
        
        \begin{enumerate}[label={\textbf{\alph*)}}]
            \item 
            \item 
        \end{enumerate}
    \end{question}

    %------Questao16--------------------------------------------------------------------------------%

    \begin{question}
        
        \begin{enumerate}[label={\textbf{\alph*)}}]
            \item 
        \end{enumerate}
    \end{question}

    %------Questao17--------------------------------------------------------------------------------%

    \begin{question}
        
        \begin{enumerate}[label={\textbf{\alph*)}}]
            \item 
            \item 
        \end{enumerate}
    \end{question}

    %------Questao18--------------------------------------------------------------------------------%

    \begin{question}
        
        \begin{enumerate}[label={\textbf{\alph*)}}]
            \item 
            \item 
        \end{enumerate}
    \end{question}
    
    %------Questao19--------------------------------------------------------------------------------%

    \begin{question}
        
        \begin{enumerate}[label={\textbf{\alph*)}}]
            \item 
            \item
        \end{enumerate}
    \end{question}

\end{document}
