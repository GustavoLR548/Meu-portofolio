\documentclass[12pt]{article}
\usepackage[T1]{fontenc}
\usepackage[utf8]{inputenc}
\usepackage[brazil]{babel}
\usepackage{hyperref}
\usepackage[a4paper,top=3.5cm,left=3cm,right=3cm,bottom=2.5cm]{geometry}
\usepackage{pgfplots}
\usepackage{enumitem}
\usepackage{amsmath}
\usepackage{graphicx}
\pgfplotsset{compat=1.8}
\usepgfplotslibrary{statistics}

\newcounter{instn}
\setcounter{instn}{1}
\newcommand{\instnum}{\arabic{instn}}

\newcommand{\myline}[1]{
    \emph{\textbf{#1)}}
    \addtocounter{instn}{1}
}

\newenvironment{question}
 {
    \myline{\instnum} 
    }
    {
 }

%configurando identação e separação de parágrafos
\parindent 1.27cm
\parskip   6pt

\setlength\parindent{0pt}

%configurando os hyperlinks
\hypersetup{
    colorlinks=true,
    linkcolor=green,
    filecolor=magenta,      
    urlcolor=blue,
}

%títulos,autor e data
\title{\textbf{Lista 3 - Estatística}}
\author{Gustavo Lopes Rodrigues}
\date{2021}

\begin{document}
    
    \maketitle

    \section*{Respostas}

    %------Questao1--------------------------------------------------------------------------------%

    \begin{question}
       
        \begin{enumerate}[label={\textbf{\alph*)}}]
            \item 
            Amostragem é uma técnica de seleção de uma amostra ou um subconjunto de elementos em uma determinada
            população que possibilita o estudo das características de uma população.
            \item 
            Amostragem Aleatória é uma amostragem pelo qual cada elemento de tamanho igual da
            população tem a mesma chance de ser selecionado no estudo que vai ser realizada.
            \item 
            Amostra aleatória simples: Todas as amostras de mesmo tamanho são igualmente prováveis.
            Amostra sistemática: Combina um processo aleatório com um processo sistémico. Percorre toda população.
            \item 
            Cidades urbanas e Interior e estudantes estudando no setor pública e estudante de escolas privadas
            \item 
            50 casa distribuido em 5 ruas, sorteando 2 segunda casa em cada rua
            30 estudante de faculdade de relações internacionais do quinto período distribuído em 3 salas, sorteando as duas
            primeiro salas
        \end{enumerate}
    \end{question}

\end{document}
