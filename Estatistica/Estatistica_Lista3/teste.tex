 %------Questao1--------------------------------------------------------------------------------%

 \begin{question}
       
    \begin{enumerate}[label={\textbf{\alph*)}}]

        \item 
        
        Amostragem é uma técnica de seleção de uma amostra ou um subconjunto de elementos em uma determinada
        população que possibilita o estudo das características de uma população.
        \item 
        
        Amostragem Aleatória é uma amostragem pelo qual cada elemento de tamanho igual da
        população tem a mesma chance de ser selecionado no estudo que vai ser realizada.
        \item 
        
        Amostra aleatória simples: Todas as amostras de mesmo tamanho são igualmente prováveis.
        Amostra sistemática: Combina um processo aleatório com um processo sistémico. Percorre toda população.
        \item 
        
        Cidades urbanas e Interior e estudantes estudando no setor pública e estudante de escolas privadas
        \item 
        
        50 casa distribuido em 5 ruas, sorteando 2 segunda casa em cada rua
        30 estudante de faculdade de relações internacionais do quinto período distribuído em 3 salas, sorteando as duas
        primeiro salas

    \end{enumerate}
\end{question}

%------Questao2--------------------------------------------------------------------------------%

\begin{question}
    
    \begin{enumerate}[label={\textbf{\alph*)}}]

        \item
              e = $\frac{Z\_a}{2} * \frac{\sigma}{\sqrt{n}}$ 
        
              e = 2.0129 * $\frac{0.22}{\sqrt{46}}$

              e = 0.0652

        \item  
              
              \begin{formula1}
                {95}{35.2}{2.0129}{0.22}{46}{0.06529290097967679}{35.1347}{35.2652}
              \end{formula1}

    \end{enumerate}
\end{question}

%------Questao3--------------------------------------------------------------------------------%

\begin{question}
    
    \begin{enumerate}[label={\textbf{\alph*)}}]

        \item  

            \begin{formula2}
                {95}{450.95}{2.2622}{6.36}{10}{4.55}{446.4}{455.5}
            \end{formula2}
    
    \end{enumerate}
\end{question}

%------Questao4--------------------------------------------------------------------------------%

\begin{question}

    \begin{formula3} 
        {95}{0.34}{1.96}{0.66}{2500}{0.018}{0.322}{0.358}
    \end{formula3}

\end{question}

%------Questao5--------------------------------------------------------------------------------%

\begin{question}
    
    \begin{formula4} 
        {0.6}{0.4}{0.02}{1.96}{0.24}{0.000104123}{2304.966241849}{2305}
    \end{formula4}

\end{question}

%------Questao6--------------------------------------------------------------------------------%

\begin{question}
    
    \begin{enumerate}[label={\textbf{\alph*)}}]

        \item 

              \begin{formula4}
                {0.7}{0.3}{0.05}{1.645}{0.21}{0.000923864}{227.306183594}{228}
              \end{formula4}

              \begin{formula5}
                {228}{100}{22800}{328}{69.512195122}{70}
              \end{formula5}

        \item 
              
              \begin{formula6} 
                {1.96}{3}{\sqrt{1}}{5.88}{34.5744}{35}
              \end{formula6}

        \item 

             \begin{formula4}
                {0.5}{0.5}{0.01}{2.576}{0.25}{0.00001507}{6589.25016589}{16590}
             \end{formula4}

             \begin{formula5}
                {16590}{500}{8295000}{17090}{85.371562317}{486}
             \end{formula5}

    \end{enumerate}
\end{question}

%------Questao7--------------------------------------------------------------------------------%

\begin{question}
    
    \begin{enumerate}[label={\textbf{\alph*)}}]

        \item 

              \begin{formula2}
                {99}{800}{2.576}{100}{400}{12.88}{787.12}{812.88}
              \end{formula2}

        \item 
              
              \begin{formula6} 
                {1.96}{100}{7.84}{24.99}{624.5}{625}
              \end{formula6}

    \end{enumerate}
\end{question}

%------Questao8--------------------------------------------------------------------------------%

\begin{question}

    \begin{formula3} 
        {90}{0.7}{1.645}{0.3}{625}{0.03}{0.67}{0.73}
    \end{formula3}
    
\end{question}

%------Questao9--------------------------------------------------------------------------------%

\begin{question}
    
    \begin{enumerate}[label={\textbf{\alph*)}}]

        \item 
            
            \begin{formula4}
                {0.6}{0.4}{0.01}{1.282}{0.24}{0.000060845}{3944.449009779}{3945}
            \end{formula4}

            \begin{formula5}
                {3945}{100}{394500}{4045}{97.527812114}{98}
            \end{formula5}
        
        \item 
            
            \begin{formula3}
                {95}{0.55}{1.96}{0.45}{100}{0.097508769}{0.4524}{0.6475}
            \end{formula3}

    \end{enumerate}
\end{question}

%------Questao10--------------------------------------------------------------------------------%

\begin{question}
    
    \begin{enumerate}[label={\textbf{\alph*)}}]

        \item 
        
             \begin{formula3}
                {95}{0.33}{1.96}{0.67}{300}{0.053}{0.277}{0.383}
             \end{formula3}

        \item 
            
            \begin{formula4}
                {0.33}{0.67}{0.02}{1.96}{0.2211}{0.000104123}{123.450150303}{2124}
            \end{formula4}

            \begin{formula5}
                {2124}{300}{637200}{2424}{62.87128712}{263}
            \end{formula5}

    \end{enumerate}
\end{question}

%------Questao11--------------------------------------------------------------------------------%

\begin{question}
    
    \begin{enumerate}[label={\textbf{\alph*)}}]

        \item 
              \begin{formula1}
                {80}{150}{1.96}{5}{36}{1.633}{148.367}{151.633}
              \end{formula1}
        \item 
              
              \begin{formula6}
                {1.96}{5}{0.98}{10}{100}{100}
              \end{formula6}
    \end{enumerate}
\end{question}

%------Questao12--------------------------------------------------------------------------------%

\begin{question}
    
    \begin{enumerate}[label={\textbf{\alph*)}}]
        
        \item 
        
            $H_0$ : $\mu$ $\geq$ 53 

            $H_1$ : $\mu$ < 53 

            Z $\rightarrow$ 2.33 (RA)

            Z $\leftarrow$ 2.33 (RR)

            \begin{formula7}
                {45}{53}{14}{30}{-3.13}
            \end{formula7}

            P(Z $\rightarrow$ 3.13) = 0.0009 ou 0.09\%
        \item 

            \begin{formula1}
               {95}{45}{1.96}{14}{30}{5.01}{39.99}{50.01}
            \end{formula1}
            
    \end{enumerate}
\end{question}

%------Questao13--------------------------------------------------------------------------------%

\begin{question}
    
    \begin{enumerate}[label={\textbf{\alph*)}}]
        
        \item 
        
            $H_0$ : $\mu$ $\leq$ 30 

            $H_1$ : $\mu$ > 30 
        
            T > 1.6766 (RR)

            T < 1.6766 (RA)

            \begin{formula7}
                {35}{30}{11}{50}{3.21}
            \end{formula7}

            P(T > 3.21) = 0.0012 ou 0.12\%

        \item 
    
            \begin{formula2}
                {99}{35}{2.0096}{11}{50}{3.13}{31.87}{38.13}
            \end{formula2}

    \end{enumerate}
\end{question}

%------Questao14--------------------------------------------------------------------------------%

\begin{question}
    
    \begin{enumerate}[label={\textbf{\alph*)}}]
        
        \item 
        
            $H_0$ : P $\leq$ 30 

            $H_1$ : P > 30 
        
            T > 1.6766 (RR)

            T < 1.6766 (RA)

            \begin{formula8}
                {0.08}{0.1}{\sqrt{0.1 * 0.9}}{100}{-0.67}
            \end{formula8}

        \item 
    
            \begin{formula3}
                {90}{0.08}{1.645}{0.92}{100}{0.044}{0.036}{0.124}
            \end{formula3}

    \end{enumerate}
\end{question}

%------Questao15--------------------------------------------------------------------------------%

\begin{question}
    
    \begin{enumerate}[label={\textbf{\alph*)}}]
        
        \item 

            \begin{formula1}
                {99}{420}{2.576}{250}{36}{107.33}{312.67}{527.33}
            \end{formula1}

        \item 
    
            A afirmação da gerência para estar de certa forma válida já que o intervalo de 
            confiança tem máximo de 527.33

    \end{enumerate}
\end{question}

%------Questao16--------------------------------------------------------------------------------%

\begin{question}
    
    \begin{enumerate}[label={\textbf{\alph*)}}]
        
        \item 
    
            $H_0$ : $\mu$ $\leq$ 6 

            $H_1$ : $\mu$ > 6 
        
            T > 2.7181 (RA)

            T < 2.7181 (RR)

            \begin{formula7}
                {6.58}{6}{1.62}{12}{1.24}
            \end{formula7}

            Não rejeita $H_o$, ao nível alfa 1\%, não há evidências sobre o número médio de 
            acidentes no cruzamento ser superior à 6.

            P(T > 1.24) = 0.1204 ou 12.04\%

    \end{enumerate}
\end{question}

%------Questao17--------------------------------------------------------------------------------%

\begin{question}
    
    \begin{enumerate}[label={\textbf{\alph*)}}]
        
        \item 
    
            $H_0$ : $\mu$ $\leq$ 6 

            $H_1$ : $\mu$ > 6 
        
            $T_c$ : 1.282

            \begin{formula7}
                {1.55}{1.5}{0.32}{50}{1.104854346}
            \end{formula7}

        \item 
    
            \begin{formula1}
                {99}{1.55}{2.576}{0.32}{50}{0.116576452}{1.433}{1.666}
            \end{formula1}

    \end{enumerate}
\end{question}

%------Questao18--------------------------------------------------------------------------------%

\begin{question}
    
    \begin{enumerate}[label={\textbf{\alph*)}}]
        
        \item 
        
            $H_0$ : $\mu$ = 15 

            $H_1$ : $\neq$ 15 
        
            T > 2.2281 ou T < -2.2281 (RR)

            -2.2281 < T < 2.2281 (RA)

            \begin{formula7}
                {14.69}{15}{5.14}{11}{-0.20}
            \end{formula7}

            Não rejeita Ho, ao nível alfa 5\%, não há evidências sobre a média do tempo 
            levado por fumantes para desistir definitivamente ser 15 anos. 

            P(T>0,2 ou T< -0,2) = 0.8455 ou 84.55\%
        \item 
    
            \begin{formula2}
                {99}{14.69}{2.576}{5.14}{11}{3.99}{10.7}{18.68}
            \end{formula2}

    \end{enumerate}
\end{question}

%------Questao19--------------------------------------------------------------------------------%

\begin{question}
    
    \begin{enumerate}[label={\textbf{\alph*)}}]

        \item
   
            $S_{fixi}$ = 7*5*9*8*11*12*13*5 = 304

            $(S_{fixi})^2$ = $7^2$5*$9^2$*8*$11^2$*12*$13^2$*5 = 3190

            $\bar{x}$ = $\frac{304}{30}$ = 10.13 

            S = $\sqrt{\frac{30*3190-304^2}{30*29}}$ = 1.92

            \begin{formula2}
                {90}{10.13}{1.6991}{1.92}{30}{0.60}{9.53}{10.73}
            \end{formula2}

        \item 

            $H_0$ : $\mu$ = 12

            $H_1$ : $\mu$ $\neq$ 12 
        
            T > 1,6991 ou T < -1,6991 (RR)

            -1,6991 < T < 1,6991 (RA)

            \begin{formula7}
                {10.13}{12}{1.92}{30}{-5.33}
            \end{formula7}

            Rejeita Ho, ao nível alfa 10\%, há evidências para rejeitar
            a média do lixo descartado ser 12 quilos. 

            P(T>5,33 ou T< -5,33) = 0,00001 ou 0,001\%

    \end{enumerate}
\end{question}