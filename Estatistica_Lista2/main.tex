\documentclass[12pt]{article}
\usepackage[T1]{fontenc}
\usepackage[utf8]{inputenc}
\usepackage[brazil]{babel}
\usepackage{hyperref}
\usepackage[a4paper,top=3.5cm,left=3cm,right=3cm,bottom=2.5cm]{geometry}
\usepackage{pgfplots}
\pgfplotsset{compat=1.8}
\usepgfplotslibrary{statistics}

\newcounter{instn}
\setcounter{instn}{1}
\newcommand{\instnum}{\arabic{instn}}

\newcommand{\myline}[1]{
    \emph{\textbf{#1)}}
    \addtocounter{instn}{1}
}

\newenvironment{question}[1]
 {
    \myline{\instnum} \\#1\\[1ex]
    }
    {
 }

 \newenvironment{formula}[1]
 {
    #1\\
    }
    {
 }

%configurando identação e separação de parágrafos
\parindent 1.27cm
\parskip   6pt

\setlength\parindent{0pt}

%configurando os hyperlinks
\hypersetup{
    colorlinks=true,
    linkcolor=green,
    filecolor=magenta,      
    urlcolor=blue,
}

%títulos,autor e data
\title{\textbf{Lista 1 - Estatística}}
\author{Gustavo Lopes Rodrigues}
\date{2021}

\begin{document}
    
    \maketitle

    \section*{Fórmulas usadas}

    \begin{formula}{Media aritmética}
        \[ M_a = \frac{x_1 + x_2 + x_3 + .... x_n }{n} \]
    \end{formula}

    \newpage

    \section*{Respostas}

    \begin{question}{Media aritmética = \textbf{77.48}, Desvio Padrão = \textbf{8.17}, Coeficiente de variação \textbf{10.55}}
        Analisando os dados, percebe-se que a média aritmética é de 77.48, uma margem de erro de 8.17\% além 
        de um coeficiente de variação igual a 10.55\%. Tais observações sugere que 
        as cotações diárias das ações dessa empresa neste intervalo são homogêneas, pois não exite 
        uma diferenciação muito grande entre os dados.
    \end{question}


\end{document}
